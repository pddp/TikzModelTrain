%%%%%%%%%%%%%%%%%%%%%
% turnout - Switch
% three-way turnout
% 
\documentclass{article}
\usepackage[a4paper]{geometry}
\usepackage[american]{babel}
\usepackage[utf8]{inputenc}
\usepackage{modeltrain}
\usepackage{pgffor}
\usepackage{sverb}
\usepackage{listings}
\lstset{language=TeX,mathescape=true,frame=Trbl,numbers=left}
\author{Dietrich Paulus}
\title{ModelTrain}
\date{Vers. 0.1 \\ \today}
\def\Raumlaenge{400}
\def\Raumbreite{250}
\def\Example#1#2#3#4{%
 \lstinputlisting[% float,%
 	firstline=#2,%
 	lastline=#3,%
	firstnumber=#2,%
 	% frame=\frame,%
 	% index=\INDEXWORDS,%
	basicstyle=\ttfamily\small,%
	keywordstyle=\ttfamily\small,%
	commentstyle=\rm,%
	label=ex:#1,%
	caption={#4}]{#1}%
}

\begin{document}

\maketitle

What is the idea behind the following package?
\begin{itemize}
  \item I want to create a track layout for a model train.
  \item I program the layout in a sequence of steps; these steps
     correspond to the steps that I would do in reality with real (model) tracks.
  \item Programming the layout should be done with \LaTeX\ and Tikz.
\end{itemize}

\section{Examples}

Currently, the package supports only planar layouts. 
Inclinations etc.\ are planned.
The typical steps are as follows:
\begin{enumerate}
  \item We select a plane size and shape on which we want to lay the tracks.
     (We do this as a polygon in Tikz).
  \item We select a starting point for the tracks in Tikz coordinates.
  \item From the starting point on we lay tacks and connect them sequentially.
  \item We select tracks of the following types:
     \begin{enumerate}
       \item straight tracks (of varying length)
       \item curved tracks (of varying angle and radius)
       \item switches (of different types)
     \end{enumerate}
\end{enumerate}

For Märklin, the following track names are admissible:
% \makeatletter
\foreach \i in \TrackNameList{\i, }
For curves we need to use either \texttt{LCurve} or \texttt{RCurve} to indicate the direction.

We start with a small complete example without switches, as shown in the following:

\begin{verbwrite}{example1}
\documentclass{standalone}
\usepackage{modeltrain}
\begin{document}
\begin{tikzpicture}[scale=0.06]
  \draw (0,0) -- (100,0) -- (100,85) -- (0,85) -- cycle; %% frame

  \draw (40,05)    % start layouting tracks here
    \Straight{5106}  % start with a straight track
    \LCurve{5100}  % 6 curves
    \LCurve{5100}
    \LCurve{5100}
    \LCurve{5100}
    \LCurve{5100}
    \LCurve{5100}
    \Straight{5106}  % one straight track
    \Rep{6}{\LCurve{5100}} % simpler, than syntax above: 6 curves $\mbox{\label{l:rep}}$
    ;
\end{tikzpicture}
\end{document}
\end{verbwrite}

\Example{example1}{1}{100}{Simple oval}

This code produces the layout as shown in \figurename~\ref{ex:example1}.
As we can see in line \ref{l:rep}, we can use a macro $\texttt{Rep}$ to lay a number of
tracks instead of repeating them explicitly.

\begin{figure}
  \centerline{\IfFileExists{example1.pdf}{\includegraphics[width=0.7\linewidth]{example1.pdf}}{file missing}}
  \caption{Example 1}\label{f:example1}
\end{figure}

In the following we will extend this picture. We will only show the code for the tikz picture.
We add two switches, one to the left, one to the right.
As these switches have three ends, we need to specify which end to connect to the current track.
The three ends are named O (for origin), S (for straight), and C (for curve, either to the left or
to the right).
These ends of the switches are marked in \figurename~\ref{ex:example2}; 
this also shows that we can mix track layout code and tikz code freely.
When we add a switch, we also give that piece a name; in the following example we have switches W1 and W2.
Each of these switches will have an open end when I lay the first loop.

Then we continue at the open end of W1 to connect it to the open end of W2.

\Example{example2}{7}{33}{Oval with two switches}

\begin{verbwrite}{example2}
\documentclass{standalone}
\usepackage{modeltrain}
\begin{document}
\begin{tikzpicture}[scale=0.06]
  \draw (0,0) -- (100,0) -- (100,105) -- (0,105) -- cycle; %% frame

  \draw (40,05)            % start layouting tracks here
    \Straight{5106}          % start with a straight track
    \Rep{3}{\LCurve{5100}} % 6 curves
    -- +(0,0) node {O}
    \LSwitch[W1][C]{5117}  % Switch named W1
    -- +(0,0) node {C}
    \LCurve{5100}          
    \LCurve{5100} 
    \Straight{5106} 
    \LCurve{5100} 
    \LCurve{5100} 
    -- +(0,0) node {C}
    \RSwitch[W2][C]{5118} % Switch named W2
    -- +(0,0) node {O}
    \Rep{3}{\LCurve{5100}} 
    ;

    \draw (W2S) -- +(0,0) node {S} ;
    
    \RestPosAngle{W1S}  % Restore settings at label W1 direction S
    \draw (W1S)         % Start drawing at this position (open end of W1)
       -- +(0,0) node {S}
       \Rep{3}{\LCurve{5100}}
       \Straight{5106}  
       \Rep{3}{\LCurve{5100}}
      ;
     
\end{tikzpicture}
\end{document}
\end{verbwrite}

\begin{figure}
  \centerline{\IfFileExists{example2.pdf}{\includegraphics[width=0.7\linewidth]{example2.pdf}}{file missing}}
  \caption{Example 2}\label{f:example2}
\end{figure}

We now add a second parallel track as shown in \figurename~\ref{f:example3}.
We do this using the following code:

\Example{example3}{7}{41}{Two oval connected by switches}

\begin{figure}
  \centerline{\IfFileExists{example3.pdf}{\includegraphics[width=0.7\linewidth]{example3.pdf}}{file missing}}
  \caption{Example 3}\label{f:example3}
\end{figure}

\begin{verbwrite}{example3}
\documentclass{standalone}
\usepackage{modeltrain}
\begin{document}
\begin{tikzpicture}[scale=0.06]
  \draw (0,0) -- (130,0) -- (130,115) -- (0,115) -- cycle; %% frame

  \draw (46,12) coordinate (Start)    % start layouting tracks here
    \Straight{5106}               % start with a straight track
    \LSwitch[W2][S]{5202L}
    \Rep{3}{\LCurve{5100}}      % 6 curves
    \LSwitch[W1][C]{5117}
    \LCurve{5100} \LCurve{5100} % 2 curves
    \Straight{5106}
    \Straight{5106}
    \LCurve{5100} \LCurve{5100} % 2 curves
    \RSwitch[W3][C]{5118}
    \Rep{3}{\LCurve{5100}}
    ;

    \RestPosAngle{W1S}
    \draw (W1S)
    \Rep{3}{\LCurve{5100}}
    \RSwitch[W6][O]{5202R}
    \Straight{5106}
    \Rep{3}{\LCurve{5100}}
    ;

  \def\Angle{0} % start horizontally
  \draw ($(Start) - (0,7.74)$) % relative to Start, we lay a second loop
    \LSwitch[W7][O]{5202L}
    \Straight{5106}
    \Rep{3}{\LCurve{5200}} % 3 curves
    \Straight{5106}
    \Rep{3}{\LCurve{5200}} % 3 curves
    \Straight{5106}
    \RSwitch[W8][S]{5202R}
    \Rep{3}{\LCurve{5200}} % 3 curves
    \Straight{5106}
    \Rep{3}{\LCurve{5200}} % 3 curves
    ;
\end{tikzpicture}
\end{document}
\end{verbwrite}

A quite complex example is given in Lst.\ref{ex:example4}
The result is shown in \figurename~\ref{f:example4}.

\begin{verbwrite}{example4}
\documentclass{article}
\usepackage[a4paper]{geometry}
\pagestyle{empty}
\usepackage{modeltrain}
\def\Raumlaenge{400}
\def\Raumbreite{250}
\begin{document}
\begin{tikzpicture}[scale=0.040]
	\coordinate (lu) at (0,0);                      % Raum links
	\coordinate (lo) at (0,\Raumlaenge);            % Raum links
	\coordinate (ro) at (\Raumbreite,\Raumlaenge);  % Raum rechts
	\coordinate (ru) at (\Raumbreite,0);            % Raum rechts
	\coordinate (cu) at (120.5,\Raumlaenge);
	
	% \draw [name path=raum] \DebugCoord(lu) -- (lo)  -- (ro) -- \DebugCoord(ru) -- cycle;
	\draw ($(cu) + (0,10)$) node {Eisenbahn -- Otterswang};

	\draw [name path=anlage] (lu) -- (lo)  -- (ro) -- (ru) 
		-- ($(ru) - (100, 0)$) 
		-- ($(ru) - (100,-300)$) 
		-- ($(lu) - (-100,-300)$) 
		-- ($(lu) + (100,0)$) -- cycle ;

    \coordinate (Start) at (56,390);
	\draw (Start)
       \Straight{5106} \Straight{5106} 
       \LSwitch[W3][S]{5202L} 
       \Straight{5106} \Straight{5106}
       \Straight{5106} \Straight{5106} \Straight{5106} \RCurve{5200} \RCurve{5200} \RCurve{5200}
       \Straight{5109} % Korrektur 1
       \Up
       \Rep{16}{\Straight{5106}}
       \Straight{5107}
       \Rep{6}{\RCurve{5200}}
       \Straight{5106}
       \Straight{5106} \Straight{5106} \Straight{5106} \RCurve{5200} \Straight{5106} \Straight{5106}
       \Straight{5106} \Straight{5106} \Straight{5106} \LCurve{5100} \Straight{5106} \Straight{5106}
       \Straight{5106} \LCurve{5100} \LCurve{5100} \LCurve{5100} 
       \Straight{5106} \Straight{5107} \Straight{5106} 
       \LCurve{5100} \LCurve{5100} \LCurve{5100} 
       \Straight{5106} 
       \Straight{5106}
       \Straight{5106} \Straight{5106} \Straight{5106} \Straight{5106} \Straight{5106} \Straight{5106}
       \Straight{5106} \Straight{5106} \Straight{5106} \Straight{5106} \Straight{5107} \Straight{5106}
       \RCurve{5200} \RCurve{5200} \RCurve{5200} \RCurve{5200} \RCurve{5200} \RCurve{5200}
       \Straight{5106} \Straight{5106} \Straight{5107} \Straight{5106} \Straight{5106} \Straight{5106}
       \Straight{5106} \Straight{5106} \Straight{5106} \Straight{5106} \Straight{5106} \Straight{5106}
       \Straight{5106} \Straight{5106} \Straight{5106} \Straight{5106} \Straight{5106} 
       \Straight{5110} % Korrektur 1
       \RCurve{5200} \RCurve{5200} \RCurve{5200} \Straight{5108}
			 % -- cycle
  ;

  \def\Angle{00}
	\draw ($(Start) - (0,7.74)$)
       \Straight{5106} \LSwitch[W1][O]{5202L}
       \Straight{5106} \Straight{5106} \Straight{5106} \Straight{5106} \Straight{5106} \Straight{5106}
       \RCurve{5100} \RCurve{5100} \RCurve{5100} 
       \Straight{5109} % Korrektur 1
       \Straight{5106} \Straight{5106} \Straight{5106}
       \Straight{5106} \Straight{5106} \Straight{5106} \Straight{5106} \Straight{5106} \Straight{5106}
       \Straight{5106} \Straight{5106} \Straight{5106} \Straight{5106} \Straight{5106} \Straight{5106}
       \Straight{5107} \Straight{5106} \RCurve{5100} \RCurve{5100} \RCurve{5100} \RCurve{5100}
       \RCurve{5100} \RCurve{5100} \Straight{5106} \Straight{5106} \Straight{5106} \Straight{5106}
       \RCurve{5100} \Straight{5106} \Straight{5106} \Straight{5106} \Straight{5106} \Straight{5106}
       \LCurve{5200} \Straight{5106} \Straight{5106} \Straight{5106} \LCurve{5200} \LCurve{5200}
       \LCurve{5200} 
       \Straight{5106} \Straight{5106} \Straight{5107} \LCurve{5200} \LCurve{5200}
       \LCurve{5200} 
       % \Straight{5106} 
       \RSwitch[W5][O]{5118}
       \Straight{5106} \Straight{5106} \Straight{5106} \Straight{5106}
       \Straight{5106} \Straight{5106} \Straight{5106} \Straight{5106} \Straight{5106} \Straight{5106}
       \Straight{5107} \Straight{5106} \Straight{5106} \RCurve{5100} \RCurve{5100} \RCurve{5100}
       \RCurve{5100} \RCurve{5100} \RCurve{5100} \Straight{5106} \Straight{5107}
       % \Straight{5106}
       \RSwitch[W2][O]{5118}
       \Straight{5106} \Straight{5106} \Straight{5106} \Straight{5106} \Straight{5106} \Straight{5106}
       \Straight{5106} \Straight{5106} \Straight{5106} \Straight{5106} \Straight{5106} \Straight{5106}
       \Straight{5106} \Straight{5106} 
       \Straight{5110} % Korrektur 1
       \RCurve{5100} \RCurve{5100} \RCurve{5100} \Straight{5108}
			 % -- cycle
	;

    % \draw (W1) node {Switch};
   \RestPosAngle{W2C}
   \draw (W2C)
        \Straight{5106} 
        \Straight{5106} 
        \LSwitch[W4][C]{5202L} 
        \Straight{5106} 
        \Straight{5106} 
        \LCurve{5205}
        \Rep{3}{\Straight{5106}}
        \LSwitch[W6][C]{5202L} 
        \RSwitch[W7][O]{5118}
        \Straight{5106} 
        \RCurve{5100}
        \Straight{5106} 
        \RCurve{5100}
        \RSwitch[W8][C]{5118} 
        \Straight{5106} 
        ;

   \RestPosAngle{W7C}
        \Rep{3}{\Straight{5106}}
        \Straight{5106} 
        \RCurve{5100}
        +(0,0) node {hier}
        ;

   \RestPosAngle{W4S}
   \draw (W4S)
        \Straight{5106} 
        \Straight{5106} 
        \Straight{5107} 
        \Straight{5109} 
        \Straight{5109} 
        \LCurve{5100} 
        \Straight{5106} 
        \Straight{5106} 
        \Straight{5108} 
        \Straight{5109} 
        \Straight{5109} 
        \Straight{5109} 
        \Straight{5109} 
        ;

   \RestPosAngle{W5C}
   \draw (W5C)
        \LCurve{5100} 
        ;
\end{tikzpicture}
\newpage
\TrackUse
\end{document}
\end{verbwrite}

\Example{example4}{1}{241}{Complex example}

\begin{figure}
  \centerline{\IfFileExists{example4.pdf}{\includegraphics[page=1,width=0.7\linewidth]{example4.pdf}}{file missing}}
  \caption{Example 4}\label{f:example4}
\end{figure}

There is also a macro that tells you the tracks that are used in a layout.
This is also used in Lst~\ref{ex:example4} and the table is shown 
\figurename~\ref{tab:example4}.
% For the layout in \figurename~\ref{f:example4} this results in a table that is shown 
% in \tablename~\ref{t:example4}.
% 
% \begin{table}
%   \centerline{\TrackUse}
%   \caption{Tracks used in \figurename~\ref{f:example4}}
% \end{table}
\begin{figure}
  \centerline{\IfFileExists{example4.pdf}{\includegraphics[page=2,width=0.7\linewidth]{example4.pdf}}{file missing}}
  \caption{Tracks used in Example 4}\label{tab:example4}
\end{figure}

\section{Known Problems}

\begin{itemize}
  \item Error handling is poor.
  \item Switches have mandatory arguments with syntax for optional arguments.
     The reason is, that these arguments will become optional eventually.
  \item More documentation required.
  \item Up to now, only planar layout
  \item Track types missing for Märklin (X-switch, 3 way turnover, ...)
  \item Support for other manufactureres missing (Fleischmann, Arno, ...)
  \item Configuration section currently writes definition to auxiliary file, as I did not get
      the \TeX\ expansion of the macros right -- should be solved more elegantly
\end{itemize}
\end{document}



\begin{center}
\begin{tikzpicture}[scale=0.060]
	\coordinate (lu) at (0,0);                      % Raum links
	\coordinate (lo) at (0,\Raumlaenge);            % Raum links
	\coordinate (ro) at (\Raumbreite,\Raumlaenge);  % Raum rechts
	\coordinate (ru) at (\Raumbreite,0);            % Raum rechts
	\coordinate (cu) at (120.5,\Raumlaenge);
	
	% \draw [name path=raum] \DebugCoord(lu) -- (lo)  -- (ro) -- \DebugCoord(ru) -- cycle;
	\draw ($(cu) + (0,10)$) node {Eisenbahn -- Otterswang};

	\draw [name path=anlage] (lu) -- (lo)  -- (ro) -- (ru) 
		-- ($(ru) - (100, 0)$) 
		-- ($(ru) - (100,-300)$) 
		-- ($(lu) - (-100,-300)$) 
		-- ($(lu) + (100,0)$) -- cycle ;

    \coordinate (Start) at (56,390);
	\draw (Start)
       \Straight{5106} \Straight{5106} 
       \LSwitch[W3][S]{5202L} 
       \Straight{5106} \Straight{5106}
       \Straight{5106} \Straight{5106} \Straight{5106} \RCurve{5200} \RCurve{5200} \RCurve{5200}
       \Straight{5109} % Korrektur 1
       \Up
       \Rep{16}{\Straight{5106}}
       \Straight{5107}
       \Rep{6}{\RCurve{5200}}
       \Straight{5106}
       \Straight{5106} \Straight{5106} \Straight{5106} \RCurve{5200} \Straight{5106} \Straight{5106}
       \Straight{5106} \Straight{5106} \Straight{5106} \LCurve{5100} \Straight{5106} \Straight{5106}
       \Straight{5106} \LCurve{5100} \LCurve{5100} \LCurve{5100} 
       \Straight{5106} \Straight{5107} \Straight{5106} 
       \LCurve{5100} \LCurve{5100} \LCurve{5100} 
       \Straight{5106} 
       \Straight{5106}
       \Straight{5106} \Straight{5106} \Straight{5106} \Straight{5106} \Straight{5106} \Straight{5106}
       \Straight{5106} \Straight{5106} \Straight{5106} \Straight{5106} \Straight{5107} \Straight{5106}
       \RCurve{5200} \RCurve{5200} \RCurve{5200} \RCurve{5200} \RCurve{5200} \RCurve{5200}
       \Straight{5106} \Straight{5106} \Straight{5107} \Straight{5106} \Straight{5106} \Straight{5106}
       \Straight{5106} \Straight{5106} \Straight{5106} \Straight{5106} \Straight{5106} \Straight{5106}
       \Straight{5106} \Straight{5106} \Straight{5106} \Straight{5106} \Straight{5106} 
       \Straight{5110} % Korrektur 1
       \RCurve{5200} \RCurve{5200} \RCurve{5200} \Straight{5108}
			 % -- cycle
  ;

  \def\Angle{00}
	\draw ($(Start) - (0,7.74)$)
       \Straight{5106} \LSwitch[W1][O]{5202L}
       \Straight{5106} \Straight{5106} \Straight{5106} \Straight{5106} \Straight{5106} \Straight{5106}
       \RCurve{5100} \RCurve{5100} \RCurve{5100} 
       \Straight{5109} % Korrektur 1
       \Straight{5106} \Straight{5106} \Straight{5106}
       \Straight{5106} \Straight{5106} \Straight{5106} \Straight{5106} \Straight{5106} \Straight{5106}
       \Straight{5106} \Straight{5106} \Straight{5106} \Straight{5106} \Straight{5106} \Straight{5106}
       \Straight{5107} \Straight{5106} \RCurve{5100} \RCurve{5100} \RCurve{5100} \RCurve{5100}
       \RCurve{5100} \RCurve{5100} \Straight{5106} \Straight{5106} \Straight{5106} \Straight{5106}
       \RCurve{5100} \Straight{5106} \Straight{5106} \Straight{5106} \Straight{5106} \Straight{5106}
       \LCurve{5200} \Straight{5106} \Straight{5106} \Straight{5106} \LCurve{5200} \LCurve{5200}
       \LCurve{5200} 
       \Straight{5106} \Straight{5106} \Straight{5107} \LCurve{5200} \LCurve{5200}
       \LCurve{5200} 
       % \Straight{5106} 
       \RSwitch[W5][O]{5118}
       \Straight{5106} \Straight{5106} \Straight{5106} \Straight{5106}
       \Straight{5106} \Straight{5106} \Straight{5106} \Straight{5106} \Straight{5106} \Straight{5106}
       \Straight{5107} \Straight{5106} \Straight{5106} \RCurve{5100} \RCurve{5100} \RCurve{5100}
       \RCurve{5100} \RCurve{5100} \RCurve{5100} \Straight{5106} \Straight{5107}
       % \Straight{5106}
       \RSwitch[W2][O]{5118}
       \Straight{5106} \Straight{5106} \Straight{5106} \Straight{5106} \Straight{5106} \Straight{5106}
       \Straight{5106} \Straight{5106} \Straight{5106} \Straight{5106} \Straight{5106} \Straight{5106}
       \Straight{5106} \Straight{5106} 
       \Straight{5110} % Korrektur 1
       \RCurve{5100} \RCurve{5100} \RCurve{5100} \Straight{5108}
			 % -- cycle
	;

    % \draw (W1) node {Switch};
   \RestPosAngle{W2C}
   \draw (W2C)
        \Straight{5106} 
        \Straight{5106} 
        \LSwitch[W4][C]{5202L} 
        \Straight{5106} 
        \Straight{5106} 
        \LCurve{5205}
        \Rep{3}{\Straight{5106}}
        \LSwitch[W6][C]{5202L} 
        \RSwitch[W7][O]{5118}
        \Straight{5106} 
        \RCurve{5100}
        \Straight{5106} 
        \RCurve{5100}
        \RSwitch[W8][C]{5118} 
        \Straight{5106} 
        ;

   \RestPosAngle{W7C}
        \Rep{3}{\Straight{5106}}
        \Straight{5106} 
        \RCurve{5100}
        +(0,0) node {hier}
        ;

   \RestPosAngle{W4S}
   \draw (W4S)
        \Straight{5106} 
        \Straight{5106} 
        \Straight{5107} 
        \Straight{5109} 
        \Straight{5109} 
        \LCurve{5100} 
        \Straight{5106} 
        \Straight{5106} 
        \Straight{5108} 
        \Straight{5109} 
        \Straight{5109} 
        \Straight{5109} 
        \Straight{5109} 
        ;

   \RestPosAngle{W5C}
   \draw (W5C)
        \LCurve{5100} 
        ;
\end{tikzpicture}
\end{center}

\newpage
\makeatletter
\textbf{Gleisbedarf}

\def\GleisAnzahlTab#1{ #1 & \@nameuse{D#1}       & \@nameuse{Dim#1} & \@nameuse{N#1}}
\begin{tabular}{llcr}
  Nummer & Beschreibung & Länge / Winkel & Anzahl \\
  \hline
  \GleisAnzahlTab{5106} \\
  \GleisAnzahlTab{5107} \\
  \GleisAnzahlTab{5108} \\
  \GleisAnzahlTab{5109} \\
  \GleisAnzahlTab{5110} \\
  % 5106  & \@nameuse{D5106}       & 180   mm & \@nameuse{N5106} \\
  % 5106  & Straight 1/1             & 180   mm & \@nameuse{N5106} \\
  % 5107  & Straight 1/2             &  90   mm & \@nameuse{N5107} \\
  % 5108  & Straight 1/4             &  45   mm & \@nameuse{N5108} \\
  % 5109  & Straight 3/16            &  33,5 mm & \@nameuse{N5107} \\
  % 5110  & Straight 1/8             &  22,5 mm & \@nameuse{N5108} \\
  \GleisAnzahlTab{5100} \\
  \GleisAnzahlTab{5147} \\
  \GleisAnzahlTab{5200} \\
  \GleisAnzahlTab{5205} \\
  \GleisAnzahlTab{5117} \\
  \GleisAnzahlTab{5118} \\
  \GleisAnzahlTab{5202L} \\
  % 5100  & Normalkreis 360 mm     & 30       & \@nameuse{N5100} \\
  % 5147  & Normalkreis 360 mm     & 15       & \@nameuse{N5147} \\
  % 5200  & Parallelkreis 437,4 mm & 30       & \@nameuse{N5200} \\
  % 5205  & Parallelkreis 437,4 mm & 5.7      & \@nameuse{N5205} \\
  % 5117  & Switch Normalkreis     & 30       & \@nameuse{N5117} \\
  % 5118  & Switch Normalkreis     & 30       & \@nameuse{N5118} \\
  % 5202  & Switch                 & 24,3     & \@nameuse{N5202} \\
  \hline
\end{tabular}

Raum: 

Winkel Dach 82 Grad

Breite Raum 370

Raum Länge 440

Tür 70

Strom: 80 vom Glas, 100 vom Dach

Gleise 5106: 150

Gleise 5107: 8 

Gleise 5100: 80

Gleise 5200: 37

Gleise 5120: 15

3 Dreiwegweichen

2 Bogenweichen links lang

1 Bogenweichen rechts lang

2 Kreuzungsweichen 5207

1 5203 L

1 512RL X

6 5119 R

1 5118 L

4 5204 R

3 5118 L

5 Prellböcke

\url{http://www.drahtkupplung.de/gtbhb2427.html}

\newpage
% \includegraphics[angle=90,width=1.0\linewidth,page=4]{0700.pdf} \\
% 
% \includegraphics[angle=90,width=1.0\linewidth,page=5]{0700.pdf} \\
% 
% \includegraphics[angle=90,width=1.0\linewidth,page=6]{0700.pdf} \\
% 
% \includegraphics[angle=90,width=1.0\linewidth,page=7]{0700.pdf} \\
% \includegraphics[width=\linewidth,page=8]{0700.pdf}
\newpage
\begin{tikzpicture}[scale=0.10]
   \draw (43.74,0) coordinate (C) arc [radius = 43.74, start angle=0, end angle=90] ;
   \draw (43.74,0) arc [radius = 36.00, start angle=0, end angle=30] ;
   \draw (36.00,0) arc [radius = 36.00, start angle=0, end angle=90] ;
\end{tikzpicture}
% \layout
\end{document}
