%%%%%%%%%%%%%%%%%%%%%
% turnout - Weiche
% three-way turnout
% 
\documentclass{article}
\usepackage[a4paper]{geometry}
\usepackage[american]{babel}
\usepackage[utf8]{inputenc}
\usepackage{modeltrain}
\usepackage{sverb}
\usepackage{listings}
\author{Dietrich Paulus}
\title{ModelTrain}
\date{Vers. 0.1 \\ \today}
\def\Raumlaenge{400}
\def\Raumbreite{250}

\begin{document}

\maketitle

We start with a small complete example, as shown in the following:

\begin{verbwrite}{example1}
\documentclass{standalone}
\usepackage{modeltrain}
\begin{document}
\begin{tikzpicture}[scale=0.06]
  \draw (0,0) -- (100,0) -- (100,85) -- (0,85) -- cycle; %% frame

  \draw (40,05)    % start layouting tracks here
    \Gerade{5106}  % start with a straight track
    \LKurve{5100}  % 6 curves
    \LKurve{5100}
    \LKurve{5100}
    \LKurve{5100}
    \LKurve{5100}
    \LKurve{5100}
    \Gerade{5106}  % one straight track
    \Rep{6}{\LKurve{5100}} % simpler, than syntax above: 6 curves
    ;
\end{tikzpicture}
\end{document}
\end{verbwrite}

\centerline{\fbox{\lstinputlisting{example1}}}

This code produces the layout as shown in \figurename~\ref{f:example1}.

\begin{figure}
  \centerline{\IfFileExists{example1.pdf}{\includegraphics[width=0.7\linewidth]{example1.pdf}}{file missing}}
  \caption{Example 1}\label{f:example1}
\end{figure}

In the following we will extend this picture. We will only show the code for the tikz picture.
We add two switches, one to the left, one to the right.
As these switches have three ends, we need to specify which end to connect to the current track.
The three ends are named O (for origin), S (for straight), and C (for curve, either to the left or
to the right).
These ends of the switches are marked in \figurename~\ref{f:example2}; 
this also shows that we can mix track layout code and tikz code freely.

\centerline{\fbox{\lstinputlisting{example2}}}

\begin{verbwrite}{example2}
\documentclass{standalone}
\usepackage{modeltrain}
\begin{document}
\begin{tikzpicture}[scale=0.06]
  \draw (0,0) -- (100,0) -- (100,105) -- (0,105) -- cycle; %% frame

  \draw (40,05)    % start layouting tracks here
    \Gerade{5106}  % start with a straight track
    \Rep{3}{\LKurve{5100}} % 6 curves
    -- +(0,0) node {O}
    \LWeiche[W1][C]{5117}
    -- +(0,0) node {C}
    \LKurve{5100} % 6 curves
    \LKurve{5100} % 6 curves
    \Gerade{5106} % start with a straight track
    \LKurve{5100} % 6 curves
    \LKurve{5100} % 6 curves
    -- +(0,0) node {C}
    \RWeiche[W2][C]{5118}
    -- +(0,0) node {O}
    \Rep{3}{\LKurve{5100}} % simpler, than syntax above: 6 curves
    ;

    \draw (W2S) -- +(0,0) node {S} ;
    
    \RestPosAngle{W1S}  % Restore settings at label W1 direction S
    \draw (W1S)         % Start drawing at this position
       -- +(0,0) node {S}
      \LKurve{5100}  
      \LKurve{5100}  
      \LKurve{5100}  
      \Gerade{5106}  
      \LKurve{5100}  
      \LKurve{5100}  
      \LKurve{5100}  
      ;
     
\end{tikzpicture}
\end{document}
\end{verbwrite}

\begin{figure}
  \centerline{\IfFileExists{example2.pdf}{\includegraphics[width=0.7\linewidth]{example2.pdf}}{file missing}}
  \caption{Example 2}\label{f:example2}
\end{figure}

We now add a second parallel track as shown in \figurename~\ref{f:example3}.
We do this using the following code:

\centerline{\fbox{\lstinputlisting{example3}}}

\begin{figure}
  \centerline{\IfFileExists{example3.pdf}{\includegraphics[width=0.7\linewidth]{example3.pdf}}{file missing}}
  \caption{Example 3}\label{f:example3}
\end{figure}

\begin{verbwrite}{example3}
\documentclass{standalone}
\usepackage{modeltrain}
\begin{document}
\begin{tikzpicture}[scale=0.06]
  \draw (0,0) -- (130,0) -- (130,115) -- (0,115) -- cycle; %% frame

  \draw (46,12) coordinate (Start)    % start layouting tracks here
    \Gerade{5106}  % start with a straight track
    % \Gerade{5106}  
    \LWeiche[W3][S]{5202} 
    \Rep{3}{\LKurve{5100}} % 6 curves
    \LWeiche[W1][C]{5117}
    \LKurve{5100} \LKurve{5100} % 2 curves
    \Gerade{5106}  
    \Gerade{5106}  
    \LKurve{5100} \LKurve{5100} % 2 curves
    \RWeiche[W2][C]{5118}
    \Rep{3}{\LKurve{5100}} 
    ;

    \RestPosAngle{W1S}
    \draw (W1S) 
    \Rep{3}{\LKurve{5100}}
    \Gerade{5106}  
    \Gerade{5106}  
    \Rep{3}{\LKurve{5100}}
    ;

  \def\Angle{0} % start horizontally
  \draw ($(Start) - (0,7.74)$)    % start layouting second track
    \LWeiche[W3][O]{5202} 
    \Gerade{5106}  
    % \Gerade{5106}  
    \Rep{3}{\LKurve{5200}} % 3 curves
    \Gerade{5106}  
    \Rep{3}{\LKurve{5200}} % 3 curves
    \Gerade{5106}  
    \Gerade{5106}  
    \Rep{3}{\LKurve{5200}} % 3 curves
    \Gerade{5106}  
    \Rep{3}{\LKurve{5200}} % 3 curves
    ;

\end{tikzpicture}
\end{document}
\end{verbwrite}

\end{document}



\begin{center}
\begin{tikzpicture}[scale=0.060]
	\coordinate (lu) at (0,0);                      % Raum links
	\coordinate (lo) at (0,\Raumlaenge);            % Raum links
	\coordinate (ro) at (\Raumbreite,\Raumlaenge);  % Raum rechts
	\coordinate (ru) at (\Raumbreite,0);            % Raum rechts
	\coordinate (cu) at (120.5,\Raumlaenge);
	
	% \draw [name path=raum] \DebugCoord(lu) -- (lo)  -- (ro) -- \DebugCoord(ru) -- cycle;
	\draw ($(cu) + (0,10)$) node {Eisenbahn -- Otterswang};

	\draw [name path=anlage] (lu) -- (lo)  -- (ro) -- (ru) 
		-- ($(ru) - (100, 0)$) 
		-- ($(ru) - (100,-300)$) 
		-- ($(lu) - (-100,-300)$) 
		-- ($(lu) + (100,0)$) -- cycle ;

    \coordinate (Start) at (56,390);
	\draw (Start)
       \Gerade{5106} \Gerade{5106} 
       \LWeiche[W3][S]{5202} 
       \Gerade{5106} \Gerade{5106}
       \Gerade{5106} \Gerade{5106} \Gerade{5106} \RKurve{5200} \RKurve{5200} \RKurve{5200}
       \Gerade{5109} % Korrektur 1
       \Up
       \Rep{16}{\Gerade{5106}}
       \Gerade{5107}
       \Rep{6}{\RKurve{5200}}
       \Gerade{5106}
       \Gerade{5106} \Gerade{5106} \Gerade{5106} \RKurve{5200} \Gerade{5106} \Gerade{5106}
       \Gerade{5106} \Gerade{5106} \Gerade{5106} \LKurve{5100} \Gerade{5106} \Gerade{5106}
       \Gerade{5106} \LKurve{5100} \LKurve{5100} \LKurve{5100} 
       \Gerade{5106} \Gerade{5107} \Gerade{5106} 
       \LKurve{5100} \LKurve{5100} \LKurve{5100} 
       \Gerade{5106} 
       \Gerade{5106}
       \Gerade{5106} \Gerade{5106} \Gerade{5106} \Gerade{5106} \Gerade{5106} \Gerade{5106}
       \Gerade{5106} \Gerade{5106} \Gerade{5106} \Gerade{5106} \Gerade{5107} \Gerade{5106}
       \RKurve{5200} \RKurve{5200} \RKurve{5200} \RKurve{5200} \RKurve{5200} \RKurve{5200}
       \Gerade{5106} \Gerade{5106} \Gerade{5107} \Gerade{5106} \Gerade{5106} \Gerade{5106}
       \Gerade{5106} \Gerade{5106} \Gerade{5106} \Gerade{5106} \Gerade{5106} \Gerade{5106}
       \Gerade{5106} \Gerade{5106} \Gerade{5106} \Gerade{5106} \Gerade{5106} 
       \Gerade{5110} % Korrektur 1
       \RKurve{5200} \RKurve{5200} \RKurve{5200} \Gerade{5108}
			 % -- cycle
  ;

  \def\Angle{00}
	\draw ($(Start) - (0,7.74)$)
       \Gerade{5106} \LWeiche[W1][O]{5202}
       \Gerade{5106} \Gerade{5106} \Gerade{5106} \Gerade{5106} \Gerade{5106} \Gerade{5106}
       \RKurve{5100} \RKurve{5100} \RKurve{5100} 
       \Gerade{5109} % Korrektur 1
       \Gerade{5106} \Gerade{5106} \Gerade{5106}
       \Gerade{5106} \Gerade{5106} \Gerade{5106} \Gerade{5106} \Gerade{5106} \Gerade{5106}
       \Gerade{5106} \Gerade{5106} \Gerade{5106} \Gerade{5106} \Gerade{5106} \Gerade{5106}
       \Gerade{5107} \Gerade{5106} \RKurve{5100} \RKurve{5100} \RKurve{5100} \RKurve{5100}
       \RKurve{5100} \RKurve{5100} \Gerade{5106} \Gerade{5106} \Gerade{5106} \Gerade{5106}
       \RKurve{5100} \Gerade{5106} \Gerade{5106} \Gerade{5106} \Gerade{5106} \Gerade{5106}
       \LKurve{5200} \Gerade{5106} \Gerade{5106} \Gerade{5106} \LKurve{5200} \LKurve{5200}
       \LKurve{5200} 
       \Gerade{5106} \Gerade{5106} \Gerade{5107} \LKurve{5200} \LKurve{5200}
       \LKurve{5200} 
       % \Gerade{5106} 
       \RWeiche[W5][O]{5118}
       \Gerade{5106} \Gerade{5106} \Gerade{5106} \Gerade{5106}
       \Gerade{5106} \Gerade{5106} \Gerade{5106} \Gerade{5106} \Gerade{5106} \Gerade{5106}
       \Gerade{5107} \Gerade{5106} \Gerade{5106} \RKurve{5100} \RKurve{5100} \RKurve{5100}
       \RKurve{5100} \RKurve{5100} \RKurve{5100} \Gerade{5106} \Gerade{5107}
       % \Gerade{5106}
       \RWeiche[W2][O]{5118}
       \Gerade{5106} \Gerade{5106} \Gerade{5106} \Gerade{5106} \Gerade{5106} \Gerade{5106}
       \Gerade{5106} \Gerade{5106} \Gerade{5106} \Gerade{5106} \Gerade{5106} \Gerade{5106}
       \Gerade{5106} \Gerade{5106} 
       \Gerade{5110} % Korrektur 1
       \RKurve{5100} \RKurve{5100} \RKurve{5100} \Gerade{5108}
			 % -- cycle
	;

    % \draw (W1) node {Weiche};
   \RestPosAngle{W2C}
   \draw (W2C)
        \Gerade{5106} 
        \Gerade{5106} 
        \LWeiche[W4][C]{5202} 
        \Gerade{5106} 
        \Gerade{5106} 
        \LKurve{5205}
        \Rep{3}{\Gerade{5106}}
        \LWeiche[W6][C]{5202} 
        \RWeiche[W7][O]{5118}
        \Gerade{5106} 
        \RKurve{5100}
        \Gerade{5106} 
        \RKurve{5100}
        \RWeiche[W8][C]{5118} 
        \Gerade{5106} 
        ;

   \RestPosAngle{W7C}
        \Rep{3}{\Gerade{5106}}
        \Gerade{5106} 
        \RKurve{5100}
        +(0,0) node {hier}
        ;

   \RestPosAngle{W4S}
   \draw (W4S)
        \Gerade{5106} 
        \Gerade{5106} 
        \Gerade{5107} 
        \Gerade{5109} 
        \Gerade{5109} 
        \LKurve{5100} 
        \Gerade{5106} 
        \Gerade{5106} 
        \Gerade{5108} 
        \Gerade{5109} 
        \Gerade{5109} 
        \Gerade{5109} 
        \Gerade{5109} 
        ;

   \RestPosAngle{W5C}
   \draw (W5C)
        \LKurve{5100} 
        ;
\end{tikzpicture}
\end{center}

\newpage
\makeatletter
\textbf{Gleisbedarf}

\def\GleisAnzahlTab#1{ #1 & \@nameuse{D#1}       & \@nameuse{Dim#1} & \@nameuse{N#1}}
\begin{tabular}{llcr}
  Nummer & Beschreibung & Länge / Winkel & Anzahl \\
  \hline
  \GleisAnzahlTab{5106} \\
  \GleisAnzahlTab{5107} \\
  \GleisAnzahlTab{5108} \\
  \GleisAnzahlTab{5109} \\
  \GleisAnzahlTab{5110} \\
  % 5106  & \@nameuse{D5106}       & 180   mm & \@nameuse{N5106} \\
  % 5106  & Gerade 1/1             & 180   mm & \@nameuse{N5106} \\
  % 5107  & Gerade 1/2             &  90   mm & \@nameuse{N5107} \\
  % 5108  & Gerade 1/4             &  45   mm & \@nameuse{N5108} \\
  % 5109  & Gerade 3/16            &  33,5 mm & \@nameuse{N5107} \\
  % 5110  & Gerade 1/8             &  22,5 mm & \@nameuse{N5108} \\
  \GleisAnzahlTab{5100} \\
  \GleisAnzahlTab{5147} \\
  \GleisAnzahlTab{5200} \\
  \GleisAnzahlTab{5205} \\
  \GleisAnzahlTab{5117} \\
  \GleisAnzahlTab{5118} \\
  \GleisAnzahlTab{5202} \\
  % 5100  & Normalkreis 360 mm     & 30       & \@nameuse{N5100} \\
  % 5147  & Normalkreis 360 mm     & 15       & \@nameuse{N5147} \\
  % 5200  & Parallelkreis 437,4 mm & 30       & \@nameuse{N5200} \\
  % 5205  & Parallelkreis 437,4 mm & 5.7      & \@nameuse{N5205} \\
  % 5117  & Weiche Normalkreis     & 30       & \@nameuse{N5117} \\
  % 5118  & Weiche Normalkreis     & 30       & \@nameuse{N5118} \\
  % 5202  & Weiche                 & 24,3     & \@nameuse{N5202} \\
  \hline
\end{tabular}

Raum: 

Winkel Dach 82 Grad

Breite Raum 370

Raum Länge 440

Tür 70

Strom: 80 vom Glas, 100 vom Dach

Gleise 5106: 150

Gleise 5107: 8 

Gleise 5100: 80

Gleise 5200: 37

Gleise 5120: 15

3 Dreiwegweichen

2 Bogenweichen links lang

1 Bogenweichen rechts lang

2 Kreuzungsweichen 5207

1 5203 L

1 512RL X

6 5119 R

1 5118 L

4 5204 R

3 5118 L

5 Prellböcke

\url{http://www.drahtkupplung.de/gtbhb2427.html}

\newpage
% \includegraphics[angle=90,width=1.0\linewidth,page=4]{0700.pdf} \\
% 
% \includegraphics[angle=90,width=1.0\linewidth,page=5]{0700.pdf} \\
% 
% \includegraphics[angle=90,width=1.0\linewidth,page=6]{0700.pdf} \\
% 
% \includegraphics[angle=90,width=1.0\linewidth,page=7]{0700.pdf} \\
% \includegraphics[width=\linewidth,page=8]{0700.pdf}
\newpage
\begin{tikzpicture}[scale=0.10]
   \draw (43.74,0) coordinate (C) arc [radius = 43.74, start angle=0, end angle=90] ;
   \draw (43.74,0) arc [radius = 36.00, start angle=0, end angle=30] ;
   \draw (36.00,0) arc [radius = 36.00, start angle=0, end angle=90] ;
\end{tikzpicture}
% \layout
\end{document}
