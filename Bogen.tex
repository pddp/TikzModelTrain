\documentclass{article}
\usepackage[a4paper]{geometry}
\geometry{hoffset=-1cm,marginparwidth=0cm,textwidth=17cm,textheight=252mm}
\usepackage[german]{babel}
\usepackage{tikz}
\usetikzlibrary{calc,intersections}
\usepackage[utf8]{inputenc}
%%%%%%%%%%%%%%%%%%%%%%%%%%%%%%%%%%%%%%%%%%%%%%%%%
\begin{document}
\def\RadA{43.74}
\def\RadB{36.00}
\begin{tikzpicture}[scale=0.30]
   \draw[red] (\RadA,0) coordinate (C) arc [radius = 43.74, start angle=0, end angle=30] 
      coordinate (o1);
   \draw[blue] (o1) coordinate (C) arc [radius = 43.74, start angle=30, end angle=60] 
      coordinate (o2);
   \draw[line width=2pt] (o2) coordinate (C) arc [radius = 43.74, start angle=60, end angle=90] 
      coordinate (o3);
   \draw[line width=2pt] (\RadA,0) arc [radius = 36.00, start angle=0, end angle=30] 
      coordinate (m1);
   \draw[red,name path=pm1,line width=2pt] (m1) arc [radius = 36.00, start angle=30, end angle=60] 
      coordinate (m2);
   \draw[line width=2pt] (\RadB,0) arc [radius = 36.00, start angle=0, end angle=30] 
      coordinate (i1);
   \draw[line width=2pt] (i1) arc [radius = 36.00, start angle=30, end angle=60] 
      coordinate (i2);
   \draw[line width=2pt] (i2) arc [radius = 36.00, start angle=60, end angle=90] 
      coordinate (i3);
   \draw (0,0) -- (\RadA,0);
   \draw (0,0) node {x};
   \draw[name path=po1,dashed] (0,0) -- (o1);
   \draw[name path=po2,dashed] (0,0) -- (o2);
   \draw ($(\RadA,0)-(\RadB,0)$) coordinate(C2) node {x};
   \draw[green,name path=pr11,dashed] (C2) -- (m1);
   \draw[dashed] (C2) -- (m2);
   \draw[dotted,name path=ptm1] (m1) -- +(120:30);
   \draw[dotted] (o2) -- +(-30:30);
   \path[name intersections={of=po1 and ptm1}] (intersection-1) coordinate(A);

   \draw[red,line width=2pt] (A) node {} -- (m1) node {};

   \draw[red,line width=2pt] (o2) arc [radius = 36.00, start angle=60, end angle=30] 
      coordinate(o4) ;

   \draw[red,line width=2pt] (m1) node {} -- (o4) node {};

   \draw[name path=pr12,dashed] (o4)  -- +( $(C2) - (m1)$ ) coordinate (C3) node {x};  % parallele

\end{tikzpicture}
% \layout
\end{document}
