% turnout - Weiche
% three-way turnout
% 
\documentclass{article}
\usepackage[a4paper]{geometry}
\usepackage[german]{babel}
\usepackage[utf8]{inputenc}
\usepackage{translations}  % tests for language base
\usepackage{calc}
\usepackage{ifthen} 
\usepackage{pgfkeys} 
\usepackage{hyperref}
\geometry{hoffset=-1cm,marginparwidth=0cm,textwidth=17cm,textheight=252mm}
\usepackage{tikz}
\usetikzlibrary{math,calc,intersections}
\usepackage{ifthen}
\usepackage{graphicx}
\parindent=0pt
\parskip=1ex
%%%%%%%%%%%%%%%%%%%%%%%%%%%%%%%%%%%%%%%%%%%%%%%%%%%%%%%%%%%%%%%%%%
% \usepackage{etoolbox} 
% \usepackage{sverb}
% \usepackage{math}
%%%%%%%%%%%%%%%%%%%%%%%%%%%%%%%%%%%%%%%%%%%%%%%%%%%%%%%%%%%%%%%%%%
% \usepackage{layout}
% \makeatletter
% \renewcommand*{\lay@value}[2]{\strip@pt\dimexpr0.351459\dimexpr\csname#2\endcsname\relax\relax mm}
% \makeatother
%%%%%%%%%%%%%%%%%%%%%%%%%%%%%%%%%%%%%%%%%%%%%%%%%%%%%%%%%%%%%%%%%%
\def\Raumlaenge{400}
\def\Raumbreite{250}
%%%%%%%%%%%%%%%%%%%%%%%%%%%%%%%%%%%%%%%%%%%%%%%%%%%%%%%%%%%%%%%%%%%%
\makeatletter
\def\DebugNode#1{node {#1}}
\def\DebugCoord(#1){(#1) node {#1}}
\def\DebugNode#1{}
\def\DebugCoord(#1){(#1)}

\def\@namegedef#1{\global\expandafter\edef\csname #1\endcsname}

\def\Angle{00}    % global: current orientation
\def\Height{0}    % global: current height
\def\Incl{0}      % inclination

\def\Up{\NoOb{\global\def\Incl{1}}}
\def\Flat{\NoOb{\global\def\Incl{0.0}}}
\def\Down{\NoOb{\global\def\Incl{-1}}}
\def\Color{black}

\def\AddToAngle#1{%
   % \typeout{Angle = \Angle}%
   \pgfmathsetmacro\XAngle{\Angle + #1}%
   \pgfmathsetmacro\YAngle{\Angle > 360 ? \XAngle -360 : \XAngle}%
   \pgfmathsetmacro\ZAngle{\Angle < -360 ? \YAngle +360 : \YAngle}%
   \global\edef\Angle{\ZAngle}%
}

\makeatletter
\newcommand\currentcoordinate{\the\tikz@lastxsaved,\the\tikz@lastysaved}

\def\StepGleis#1{%
   \expandafter\@tempcnta\@nameuse{N#1}%
   \advance\@tempcnta by 1\relax
   \@namegedef{N#1}{\the\@tempcnta}\relax
}

\def\Bullet{\pgfmathsetmacro\XAngle{\Angle + 90}\rotatebox{\XAngle}{\tiny{|}}}

\def\Rep#1{\foreach \i in {1,2,...,#1}}

\def\Gerade#1{\@nameuse{G#1}\NoOb{\StepGleis{#1}}}
\def\LKurve#1{\@nameuse{L#1}\NoOb{\StepGleis{#1}}}
\def\RKurve#1{\@nameuse{R#1}\NoOb{\StepGleis{#1}}}
\def\LWeiche[#1][#2]#3{\@nameuse{L#2#3}{#1}\NoOb{\StepGleis{#3}}}
\def\RWeiche[#1][#2]#3{\@nameuse{R#2#3}{#1}\NoOb{\StepGleis{#3}}}

%%%%%%%%%%%%%%%%%%%%%%%%%%%%%%%%%%%%%%%%%%%%
\pgfkeys{%
  /conf/.is family,%
  /conf,%
  id/.initial = {},%
  german/.initial = {},%
  english/.initial = {},%
  type/.initial = {Straight},%
  length/.initial = {0},%
  angle/.initial = {0},%
  radius/.initial = {0},%
}
\newcommand{\Config}[1]{%
  \pgfkeys{/conf, 
    #1,
    id/.get=\Id,
    type/.get=\Type,
    german/.get=\German,
    english/.get=\English,
    angle/.get=\ConfAngle,
    radius/.get=\ConfRadius,
    length/.get=\ConfLength,
  }%
  \@namedef{N\Id}{0}\relax        % count of tracks for this id
  \immediate\write\ConfigFile{\string\@namedef{D\Id}{\string\ifcurrentbaselanguage{English}{\English}{\German}}\relax} 
  \immediate\write\ConfigFile{\string\@nameuse{Def\Type}{\Id}{\ConfLength}{\ConfRadius}{\ConfAngle}\relax}\relax
}
\def\DefStraight#1#2#3#4{\global\@namedef{G#1}{\NormalGerade{#2}}\relax}
\def\DefCurve#1#2#3#4{\global\@namedef{R#1}{\RKurvePar{#3}{#4}{#1}}\global\@namedef{L#1}{\LKurvePar{#3}{#4}{#1}}}
\def\DefSwitchL#1#2#3#4{%
    \@namedef{LO#1}##1{\LWeicheParO{##1}{#3}{#4}{#1}}%
    \@namedef{LS#1}##1{\LWeicheParS{##1}{#3}{#4}{#1}}%
    \@namedef{LC#1}##1{\LWeicheParC{##1}{#3}{#4}{#1}}%
}
\def\DefSwitchR#1#2#3#4{%
    \@namedef{RO#1}##1{\RWeicheParO{##1}{#3}{#4}{#1}}%
    \@namedef{RS#1}##1{\RWeicheParS{##1}{#3}{#4}{#1}}%
    \@namedef{RC#1}##1{\RWeicheParC{##1}{#3}{#4}{#1}}%
}

\def\NoOb{arc [radius = 50, start angle=0, end angle=0] node}

\def\SavePosAngle#1{coordinate(#1) 
    \NoOb{\@namegedef{#1-angle}{\Angle}%
          \@namegedef{#1-height}{\Height}}}

\def\RestPosAngle#1{(#1) 
    \NoOb{\global\edef\Angle{\@nameuse{#1-angle}}%
          \global\edef\Height{\@nameuse{#1-height}}}}

\def\NormalGerade#1{-- +(\Angle:#1) node {\Bullet} 
    \NoOb {%
        \ifnum\Incl>0\relax
           \pgfmathsetmacro\Xheight{1.02 * \Height}%
           \global\edef\Height{\Xheight}%
        \fi
    }
}
%%%
\def\RKurvePar#1#2#3{\KurvePar{#1}{#2}{#3}{90}}
\def\LKurvePar#1#2#3{\KurvePar{#1}{-#2}{#3}{-90}}
\def\KurvePar#1#2#3#4{%
    % Trick: hide computation in invisible arc
    \NoOb {%
        \pgfmathsetmacro\XStartA{\Angle + #4}%
        \pgfmathsetmacro\XEndA{\Angle +#4 - #2}%
        \global\edef\StartA{\XStartA}%
        \global\edef\EndA{\XEndA}%
        % \tiny{#3}%
    }
	arc [radius=#1,                % fix
         start angle=\StartA,      % Angle +  90
	 	 end angle=\EndA]          % Angle +  60
         node {\AddToAngle{-#2}\Bullet}
}
%%%%%%%%%%%%%%%%%%%%%%%%%%%%%%%%%%%%%%%%%%%%%%%%%%%%%%%%%%%%%%%%%%%%%%%%%%%%%%%%%%
%                                                                              
%   (O)rigin -------- (S)traight                       / (C)urve
%                \                                    /                                      
%                 \                                  /                                        
%                  \                                /                                         
%                   \ (C)urve            (O)rigin -------- (S)traight                   
%                                                                              
%%%%%%%%%%%%%%%%%%%%%%%%%%%%%%%%%%%%%%%%%%%%%%%%%%%%%%%%%%%%%%%%%%%%%%%%%%%%%%%%%%

\def\LWeicheParO#1#2#3#4{%
   \SavePosAngle{#1}%
   \LKurvePar{#2}{#3}{#4}{}%
   \SavePosAngle{#1S}%
   \SavePosAngle{#1C}%
   \RestPosAngle{#1}%
   \NormalGerade{18}%
}
\def\LWeicheParS#1#2#3#4{%
   \NormalGerade{18}%
   \SavePosAngle{#1}%
   \SavePosAngle{#1S}%
   \NoOb {\AddToAngle{-180}}
   \LKurvePar{#2}{#3}{#4}{}%
   \SavePosAngle{#1C}%
   \RestPosAngle{#1}%
}
\def\LWeicheParC#1#2#3#4{%
   \SavePosAngle{#1}%
   \NormalGerade{18}%
   \SavePosAngle{#1S}%
   \RestPosAngle{#1}%
   \LKurvePar{#2}{#3}{#4}{}%
}
\def\RWeicheParS#1#2#3#4{%
   \NormalGerade{18}%
   \SavePosAngle{#1S}%
   \NoOb {\AddToAngle{-180}}
   \RKurvePar{#2}{#3}{#4}{}%
   \SavePosAngle{#1C}%
   \RestPosAngle{#1S}%
}
\def\RWeicheParC#1#2#3#4{%
   \SavePosAngle{#1}%
   \RKurvePar{#2}{#3}{#4}{}%
   \SavePosAngle{#1C}%
   \RestPosAngle{#1}%
   \NormalGerade{18}%
   \SavePosAngle{#1S}%
}
\def\RWeicheParO#1#2#3#4{%
   \SavePosAngle{#1}%
   \RKurvePar{#2}{#3}{#4}{}%
   \SavePosAngle{#1C}%
   \RestPosAngle{#1}%
   \NormalGerade{18}%
   \SavePosAngle{#1S}%
}
%%%%%%%%%%%%%%%%%%%%%%%%%%%%%%%%%%%%%%%%%%%%
% Märklin Gleise
%%%%%%%%%%%%%%%%%%%%%%%%%%%%%%%%%%%%%%%%%%%%
\newwrite\ConfigFile
\immediate\openout\ConfigFile=\jobname.cfg
\Config{id=5106,type={Straight},german={Gerade 1/1},english={straight track 1/1},length={18}}
\Config{id=5107,type={Straight},german={Gerade 1/2},english={straight track 1/2},length={9}}
\Config{id=5108,type={Straight},german={Gerade 1/4},english={straight track 1/4},length={4.5}}
\Config{id=5109,type={Straight},german={Gerade 3/16},english={straight track 3/16},length={3.35}}
\Config{id=5110,type={Straight},german={Gerade 1/8},english={straight track 1/8},length={2.25}}
\Config{id=5100,type={Curve},german={Normalkreis},english={normal curve},radius={36.0},angle={30}}
\Config{id=5147,type={Curve},german={Normalkreis},english={normal curve},radius={36.0},angle={15}}
\Config{id=5200,type={Curve},german={Aussenkreis},english={outer curve},radius={43.74},angle={30}}  % ß!
\Config{id=5205,type={Curve},german={Aussenkreis},english={outer curve},radius={43.74},angle={5.7}} % ß!
%%%%%%%%%%
\Config{id=5117,type={SwitchL},german={Weiche links},english={track switch left},radius={36.00},angle={30}}
\Config{id=5118,type={SwitchR},german={Weiche rechts},english={track switch right},radius={36.00},angle={30}}
\Config{id=5141,type={SwitchL},german={Bogenweiche links},english={curved track switch left},
        radius={36.00},angle={30}}
\Config{id=5142,type={SwitchR},german={Bogenweiche rechts},english={curved track switch right},
        radius={36.00},angle={30}}
\Config{id=5202,type={SwitchL},german={Weiche},english={track switch},radius={43.74},angle={24.3}}
\immediate\closeout\ConfigFile
\input{\jobname.cfg}
%%%%%%%%%%%%%%%%%%%%%%%%%%%%%%%%%%%%%%%%%%%%

%%%%%%%%%%%%%%%%%%%%%%%%%%%%%%%%%%%%%%%%%%%%%%%%%
\begin{document}
\thispagestyle{empty}
\begin{center}
\begin{tikzpicture}[scale=0.060]
	\coordinate (lu) at (0,0);                      % Raum links
	\coordinate (lo) at (0,\Raumlaenge);            % Raum links
	\coordinate (ro) at (\Raumbreite,\Raumlaenge);  % Raum rechts
	\coordinate (ru) at (\Raumbreite,0);            % Raum rechts
	\coordinate (cu) at (120.5,\Raumlaenge);
	
	% \draw [name path=raum] \DebugCoord(lu) -- (lo)  -- (ro) -- \DebugCoord(ru) -- cycle;
	\draw ($(cu) + (0,10)$) node {Eisenbahn -- Otterswang};

	\draw [name path=anlage] (lu) -- (lo)  -- (ro) -- (ru) 
		-- ($(ru) - (100, 0)$) 
		-- ($(ru) - (100,-300)$) 
		-- ($(lu) - (-100,-300)$) 
		-- ($(lu) + (100,0)$) -- cycle ;

    \coordinate (Start) at (56,390);
	\draw (Start)
       \Gerade{5106} \Gerade{5106} 
       \LWeiche[W3][S]{5202} 
       \Gerade{5106} \Gerade{5106}
       \Gerade{5106} \Gerade{5106} \Gerade{5106} \RKurve{5200} \RKurve{5200} \RKurve{5200}
       \Gerade{5109} % Korrektur 1
       \Up
       \Rep{16}{\Gerade{5106}}
       \Gerade{5107}
       \Rep{6}{\RKurve{5200}}
       \Gerade{5106}
       \Gerade{5106} \Gerade{5106} \Gerade{5106} \RKurve{5200} \Gerade{5106} \Gerade{5106}
       \Gerade{5106} \Gerade{5106} \Gerade{5106} \LKurve{5100} \Gerade{5106} \Gerade{5106}
       \Gerade{5106} \LKurve{5100} \LKurve{5100} \LKurve{5100} 
       \Gerade{5106} \Gerade{5107} \Gerade{5106} 
       \LKurve{5100} \LKurve{5100} \LKurve{5100} 
       \Gerade{5106} 
       \Gerade{5106}
       \Gerade{5106} \Gerade{5106} \Gerade{5106} \Gerade{5106} \Gerade{5106} \Gerade{5106}
       \Gerade{5106} \Gerade{5106} \Gerade{5106} \Gerade{5106} \Gerade{5107} \Gerade{5106}
       \RKurve{5200} \RKurve{5200} \RKurve{5200} \RKurve{5200} \RKurve{5200} \RKurve{5200}
       \Gerade{5106} \Gerade{5106} \Gerade{5107} \Gerade{5106} \Gerade{5106} \Gerade{5106}
       \Gerade{5106} \Gerade{5106} \Gerade{5106} \Gerade{5106} \Gerade{5106} \Gerade{5106}
       \Gerade{5106} \Gerade{5106} \Gerade{5106} \Gerade{5106} \Gerade{5106} 
       \Gerade{5110} % Korrektur 1
       \RKurve{5200} \RKurve{5200} \RKurve{5200} \Gerade{5108}
			 % -- cycle
  ;

  \def\Angle{00}
	\draw ($(Start) - (0,7.74)$)
       \Gerade{5106} \LWeiche[W1][O]{5202}
       \Gerade{5106} \Gerade{5106} \Gerade{5106} \Gerade{5106} \Gerade{5106} \Gerade{5106}
       \RKurve{5100} \RKurve{5100} \RKurve{5100} 
       \Gerade{5109} % Korrektur 1
       \Gerade{5106} \Gerade{5106} \Gerade{5106}
       \Gerade{5106} \Gerade{5106} \Gerade{5106} \Gerade{5106} \Gerade{5106} \Gerade{5106}
       \Gerade{5106} \Gerade{5106} \Gerade{5106} \Gerade{5106} \Gerade{5106} \Gerade{5106}
       \Gerade{5107} \Gerade{5106} \RKurve{5100} \RKurve{5100} \RKurve{5100} \RKurve{5100}
       \RKurve{5100} \RKurve{5100} \Gerade{5106} \Gerade{5106} \Gerade{5106} \Gerade{5106}
       \RKurve{5100} \Gerade{5106} \Gerade{5106} \Gerade{5106} \Gerade{5106} \Gerade{5106}
       \LKurve{5200} \Gerade{5106} \Gerade{5106} \Gerade{5106} \LKurve{5200} \LKurve{5200}
       \LKurve{5200} 
       \Gerade{5106} \Gerade{5106} \Gerade{5107} \LKurve{5200} \LKurve{5200}
       \LKurve{5200} 
       % \Gerade{5106} 
       \RWeiche[W5][O]{5118}
       \Gerade{5106} \Gerade{5106} \Gerade{5106} \Gerade{5106}
       \Gerade{5106} \Gerade{5106} \Gerade{5106} \Gerade{5106} \Gerade{5106} \Gerade{5106}
       \Gerade{5107} \Gerade{5106} \Gerade{5106} \RKurve{5100} \RKurve{5100} \RKurve{5100}
       \RKurve{5100} \RKurve{5100} \RKurve{5100} \Gerade{5106} \Gerade{5107}
       % \Gerade{5106}
       \RWeiche[W2][O]{5118}
       \Gerade{5106} \Gerade{5106} \Gerade{5106} \Gerade{5106} \Gerade{5106} \Gerade{5106}
       \Gerade{5106} \Gerade{5106} \Gerade{5106} \Gerade{5106} \Gerade{5106} \Gerade{5106}
       \Gerade{5106} \Gerade{5106} 
       \Gerade{5110} % Korrektur 1
       \RKurve{5100} \RKurve{5100} \RKurve{5100} \Gerade{5108}
			 % -- cycle
	;

    % \draw (W1) node {Weiche};
   \RestPosAngle{W2C}
   \draw (W2C)
        \Gerade{5106} 
        \Gerade{5106} 
        \LWeiche[W4][C]{5202} 
        \Gerade{5106} 
        \Gerade{5106} 
        \LKurve{5205}
        \Rep{3}{\Gerade{5106}}
        \LWeiche[W6][C]{5202} 
        \RWeiche[W7][O]{5118}
        \Gerade{5106} 
        \RKurve{5100}
        \Gerade{5106} 
        \RKurve{5100}
        \RWeiche[W8][C]{5118} 
        \Gerade{5106} 
        ;

   \RestPosAngle{W7C}
        \Rep{3}{\Gerade{5106}}
        \Gerade{5106} 
        \RKurve{5100}
        +(0,0) node {hier}
        ;

   \RestPosAngle{W4S}
   \draw (W4S)
        \Gerade{5106} 
        \Gerade{5106} 
        \Gerade{5107} 
        \Gerade{5109} 
        \Gerade{5109} 
        \LKurve{5100} 
        \Gerade{5106} 
        \Gerade{5106} 
        \Gerade{5108} 
        \Gerade{5109} 
        \Gerade{5109} 
        \Gerade{5109} 
        \Gerade{5109} 
        ;

   \RestPosAngle{W5C}
   \draw (W5C)
        \LKurve{5100} 
        ;
\end{tikzpicture}
\end{center}

\newpage
\textbf{Gleisbedarf}

\def\GleisAnzahlTab#1{ #1 & \@nameuse{D#1}       & \@nameuse{Dim#1} & \@nameuse{N#1}}
\begin{tabular}{llcr}
  Nummer & Beschreibung & Länge / Winkel & Anzahl \\
  \hline
  \GleisAnzahlTab{5106} \\
  \GleisAnzahlTab{5107} \\
  \GleisAnzahlTab{5108} \\
  \GleisAnzahlTab{5109} \\
  \GleisAnzahlTab{5110} \\
  % 5106  & \@nameuse{D5106}       & 180   mm & \@nameuse{N5106} \\
  % 5106  & Gerade 1/1             & 180   mm & \@nameuse{N5106} \\
  % 5107  & Gerade 1/2             &  90   mm & \@nameuse{N5107} \\
  % 5108  & Gerade 1/4             &  45   mm & \@nameuse{N5108} \\
  % 5109  & Gerade 3/16            &  33,5 mm & \@nameuse{N5107} \\
  % 5110  & Gerade 1/8             &  22,5 mm & \@nameuse{N5108} \\
  \GleisAnzahlTab{5100} \\
  \GleisAnzahlTab{5147} \\
  \GleisAnzahlTab{5200} \\
  \GleisAnzahlTab{5205} \\
  \GleisAnzahlTab{5117} \\
  \GleisAnzahlTab{5118} \\
  \GleisAnzahlTab{5202} \\
  % 5100  & Normalkreis 360 mm     & 30       & \@nameuse{N5100} \\
  % 5147  & Normalkreis 360 mm     & 15       & \@nameuse{N5147} \\
  % 5200  & Parallelkreis 437,4 mm & 30       & \@nameuse{N5200} \\
  % 5205  & Parallelkreis 437,4 mm & 5.7      & \@nameuse{N5205} \\
  % 5117  & Weiche Normalkreis     & 30       & \@nameuse{N5117} \\
  % 5118  & Weiche Normalkreis     & 30       & \@nameuse{N5118} \\
  % 5202  & Weiche                 & 24,3     & \@nameuse{N5202} \\
  \hline
\end{tabular}

Raum: 

Winkel Dach 82 Grad

Breite Raum 370

Raum Länge 440

Tür 70

Strom: 80 vom Glas, 100 vom Dach

Gleise 5106: 150

Gleise 5107: 8 

Gleise 5100: 80

Gleise 5200: 37

Gleise 5120: 15

3 Dreiwegweichen

2 Bogenweichen links lang

1 Bogenweichen rechts lang

2 Kreuzungsweichen 5207

1 5203 L

1 512RL X

6 5119 R

1 5118 L

4 5204 R

3 5118 L

5 Prellböcke

\url{http://www.drahtkupplung.de/gtbhb2427.html}

\newpage
% \includegraphics[angle=90,width=1.0\linewidth,page=4]{0700.pdf} \\
% 
% \includegraphics[angle=90,width=1.0\linewidth,page=5]{0700.pdf} \\
% 
% \includegraphics[angle=90,width=1.0\linewidth,page=6]{0700.pdf} \\
% 
% \includegraphics[angle=90,width=1.0\linewidth,page=7]{0700.pdf} \\
% \includegraphics[width=\linewidth,page=8]{0700.pdf}
% \newpage
% \layout
\end{document}
