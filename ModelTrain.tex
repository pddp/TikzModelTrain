% turnout - Switch
% three-way turnout
% 
\documentclass{article}
\usepackage[a4paper]{geometry}
\geometry{hoffset=-1cm,marginparwidth=0cm,textwidth=17cm,textheight=252mm}
\usepackage[german]{babel}
\usepackage[utf8]{inputenc}
\usepackage{modeltrain}
%%%%%%%%%%%%%%%%%%%%%%%%%%%%%%%%%%%%%%%%%%%%%%%%%%%%%%%%%%%%%%%%%%
% \usepackage{layout}
% \makeatletter
% \renewcommand*{\lay@value}[2]{\strip@pt\dimexpr0.351459\dimexpr\csname#2\endcsname\relax\relax mm}
% \makeatother
%%%%%%%%%%%%%%%%%%%%%%%%%%%%%%%%%%%%%%%%%%%%%%%%%%%%%%%%%%%%%%%%%%
\def\Raumlaenge{400}
\def\Raumbreite{250}
%%%%%%%%%%%%%%%%%%%%%%%%%%%%%%%%%%%%%%%%%%%%%%%%%
\begin{document}
\thispagestyle{empty}
\begin{center}
\begin{tikzpicture}[scale=0.060]
	\coordinate (lu) at (0,0);                      % Raum links
	\coordinate (lo) at (0,\Raumlaenge);            % Raum links
	\coordinate (ro) at (\Raumbreite,\Raumlaenge);  % Raum rechts
	\coordinate (ru) at (\Raumbreite,0);            % Raum rechts
	\coordinate (cu) at (120.5,\Raumlaenge);
	
	% \draw [name path=raum] \DebugCoord(lu) -- (lo)  -- (ro) -- \DebugCoord(ru) -- cycle;
	\draw ($(cu) + (0,10)$) node {Eisenbahn -- Otterswang};

	\draw [name path=anlage] (lu) -- (lo)  -- (ro) -- (ru) 
		-- ($(ru) - (100, 0)$) 
		-- ($(ru) - (100,-300)$) 
		-- ($(lu) - (-100,-300)$) 
		-- ($(lu) + (100,0)$) -- cycle ;

    \coordinate (Start) at (56,390);
	\draw (Start)
       \Straight{5106} \Straight{5106} 
       \LSwitch[W3][S]{5202L} 
       \Straight{5106} \Straight{5106}
       \Straight{5106} \Straight{5106} \Straight{5106} \RCurve{5200} \RCurve{5200} \RCurve{5200}
       \Straight{5109} % Korrektur 1
       \Up
       \Rep{16}{\Straight{5106}}
       \Straight{5107}
       \Rep{6}{\RCurve{5200}}
       \Straight{5106}
       \Straight{5106} \Straight{5106} \Straight{5106} \RCurve{5200} \Straight{5106} \Straight{5106}
       \Straight{5106} \Straight{5106} \Straight{5106} \LCurve{5100} \Straight{5106} \Straight{5106}
       \Straight{5106} \LCurve{5100} \LCurve{5100} \LCurve{5100} 
       \Straight{5106} \Straight{5107} \Straight{5106} 
       \LCurve{5100} \LCurve{5100} \LCurve{5100} 
       \Straight{5106} 
       \Straight{5106}
       \Straight{5106} \Straight{5106} \Straight{5106} \Straight{5106} \Straight{5106} \Straight{5106}
       \Straight{5106} \Straight{5106} \Straight{5106} \Straight{5106} \Straight{5107} \Straight{5106}
       \RCurve{5200} \RCurve{5200} \RCurve{5200} \RCurve{5200} \RCurve{5200} \RCurve{5200}
       \Straight{5106} \Straight{5106} \Straight{5107} \Straight{5106} \Straight{5106} \Straight{5106}
       \Straight{5106} \Straight{5106} \Straight{5106} \Straight{5106} \Straight{5106} \Straight{5106}
       \Straight{5106} \Straight{5106} \Straight{5106} \Straight{5106} \Straight{5106} 
       \Straight{5110} % Korrektur 1
       \RCurve{5200} \RCurve{5200} \RCurve{5200} \Straight{5108}
			 % -- cycle
  ;

  \def\Angle{00}
	\draw ($(Start) - (0,7.74)$)
       \Straight{5106} \LSwitch[W1][O]{5202L}
       \Straight{5106} \Straight{5106} \Straight{5106} \Straight{5106} \Straight{5106} \Straight{5106}
       \RCurve{5100} \RCurve{5100} \RCurve{5100} 
       \Straight{5109} % Korrektur 1
       \Straight{5106} \Straight{5106} \Straight{5106}
       \Straight{5106} \Straight{5106} \Straight{5106} \Straight{5106} \Straight{5106} \Straight{5106}
       \Straight{5106} \Straight{5106} \Straight{5106} \Straight{5106} \Straight{5106} \Straight{5106}
       \Straight{5107} \Straight{5106} \RCurve{5100} \RCurve{5100} \RCurve{5100} \RCurve{5100}
       \RCurve{5100} \RCurve{5100} \Straight{5106} \Straight{5106} \Straight{5106} \Straight{5106}
       \RCurve{5100} \Straight{5106} \Straight{5106} \Straight{5106} \Straight{5106} \Straight{5106}
       \LCurve{5200} \Straight{5106} \Straight{5106} \Straight{5106} \LCurve{5200} \LCurve{5200}
       \LCurve{5200} 
       \Straight{5106} \Straight{5106} \Straight{5107} \LCurve{5200} \LCurve{5200}
       \LCurve{5200} 
       % \Straight{5106} 
       \RSwitch[W5][O]{5118}
       \Straight{5106} \Straight{5106} \Straight{5106} \Straight{5106}
       \Straight{5106} \Straight{5106} \Straight{5106} \Straight{5106} \Straight{5106} \Straight{5106}
       \Straight{5107} \Straight{5106} \Straight{5106} \RCurve{5100} \RCurve{5100} \RCurve{5100}
       \RCurve{5100} \RCurve{5100} \RCurve{5100} \Straight{5106} \Straight{5107}
       % \Straight{5106}
       \RSwitch[W2][O]{5118}
       \Straight{5106} \Straight{5106} \Straight{5106} \Straight{5106} \Straight{5106} \Straight{5106}
       \Straight{5106} \Straight{5106} \Straight{5106} \Straight{5106} \Straight{5106} \Straight{5106}
       \Straight{5106} \Straight{5106} 
       \Straight{5110} % Korrektur 1
       \RCurve{5100} \RCurve{5100} \RCurve{5100} \Straight{5108}
			 % -- cycle
	;

    % \draw (W1) node {Switch};
   \RestPosAngle{W2C}
   \draw (W2C)
        \Straight{5106} 
        \Straight{5106} 
        \LSwitch[W4][C]{5202L} 
        \Straight{5106} 
        \Straight{5106} 
        \LCurve{5205}
        \Rep{3}{\Straight{5106}}
        \LSwitch[W6][C]{5202L} 
        \RSwitch[W7][O]{5118}
        \Straight{5106} 
        \RCurve{5100}
        \Straight{5106} 
        \RCurve{5100}
        \RSwitch[W8][C]{5118} 
        \Straight{5106} 
        ;

   \RestPosAngle{W7C}
        \Rep{3}{\Straight{5106}}
        \Straight{5106} 
        \RCurve{5100}
        +(0,0) node {hier}
        ;

   \RestPosAngle{W4S}
   \draw (W4S)
        \Straight{5106} 
        \Straight{5106} 
        \Straight{5107} 
        \Straight{5109} 
        \Straight{5109} 
        \LCurve{5100} 
        \Straight{5106} 
        \Straight{5106} 
        \Straight{5108} 
        \Straight{5109} 
        \Straight{5109} 
        \Straight{5109} 
        \Straight{5109} 
        ;

   \RestPosAngle{W5C}
   \draw (W5C)
        \LCurve{5100} 
        ;
\end{tikzpicture}
\end{center}

\newpage
\makeatletter
\textbf{Gleisbedarf}

\def\GleisAnzahlTab#1{ #1 & \@nameuse{D#1}       & \@nameuse{Dim#1} & \@nameuse{N#1}}
\begin{tabular}{llcr}
  Nummer & Beschreibung & Länge / Winkel & Anzahl \\
  \hline
  \GleisAnzahlTab{5106} \\
  \GleisAnzahlTab{5107} \\
  \GleisAnzahlTab{5108} \\
  \GleisAnzahlTab{5109} \\
  \GleisAnzahlTab{5110} \\
  % 5106  & \@nameuse{D5106}       & 180   mm & \@nameuse{N5106} \\
  % 5106  & Straight 1/1             & 180   mm & \@nameuse{N5106} \\
  % 5107  & Straight 1/2             &  90   mm & \@nameuse{N5107} \\
  % 5108  & Straight 1/4             &  45   mm & \@nameuse{N5108} \\
  % 5109  & Straight 3/16            &  33,5 mm & \@nameuse{N5107} \\
  % 5110  & Straight 1/8             &  22,5 mm & \@nameuse{N5108} \\
  \GleisAnzahlTab{5100} \\
  \GleisAnzahlTab{5147} \\
  \GleisAnzahlTab{5200} \\
  \GleisAnzahlTab{5205} \\
  \GleisAnzahlTab{5117} \\
  \GleisAnzahlTab{5118} \\
  \GleisAnzahlTab{5202L} \\
  % 5100  & Normalkreis 360 mm     & 30       & \@nameuse{N5100} \\
  % 5147  & Normalkreis 360 mm     & 15       & \@nameuse{N5147} \\
  % 5200  & Parallelkreis 437,4 mm & 30       & \@nameuse{N5200} \\
  % 5205  & Parallelkreis 437,4 mm & 5.7      & \@nameuse{N5205} \\
  % 5117  & Switch Normalkreis     & 30       & \@nameuse{N5117} \\
  % 5118  & Switch Normalkreis     & 30       & \@nameuse{N5118} \\
  % 5202  & Switch                 & 24,3     & \@nameuse{N5202} \\
  \hline
\end{tabular}

Raum: 

Winkel Dach 82 Grad

Breite Raum 370

Raum Länge 440

Tür 70

Strom: 80 vom Glas, 100 vom Dach

Gleise 5106: 150

Gleise 5107: 8 

Gleise 5100: 80

Gleise 5200: 37

Gleise 5120: 15

3 Dreiwegweichen

2 Bogenweichen links lang

1 Bogenweichen rechts lang

2 Kreuzungsweichen 5207

1 5203 L

1 512RL X

6 5119 R

1 5118 L

4 5204 R

3 5118 L

5 Prellböcke

\url{http://www.drahtkupplung.de/gtbhb2427.html}

\newpage
% \includegraphics[angle=90,width=1.0\linewidth,page=4]{0700.pdf} \\
% 
% \includegraphics[angle=90,width=1.0\linewidth,page=5]{0700.pdf} \\
% 
% \includegraphics[angle=90,width=1.0\linewidth,page=6]{0700.pdf} \\
% 
% \includegraphics[angle=90,width=1.0\linewidth,page=7]{0700.pdf} \\
% \includegraphics[width=\linewidth,page=8]{0700.pdf}
\newpage
\begin{tikzpicture}[scale=0.10]
   \draw (43.74,0) coordinate (C) arc [radius = 43.74, start angle=0, end angle=90] ;
   \draw (43.74,0) arc [radius = 36.00, start angle=0, end angle=30] ;
   \draw (36.00,0) arc [radius = 36.00, start angle=0, end angle=90] ;
\end{tikzpicture}
% \layout
\end{document}
