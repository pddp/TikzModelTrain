%%%%%%%%%%%%%%
%
% turnout - Switch
% three-way turnout
% 
%%%%%%%%%%%%%%
\documentclass{article}
\usepackage[a4paper]{geometry}
\geometry{hoffset=-1cm,marginparwidth=0cm,textwidth=17cm,textheight=252mm}
\usepackage[german]{babel}
\usepackage[utf8]{inputenc}
\usepackage{hyperref}
\usepackage{modeltrain}
%%%%%%%%%%%%%%%%%%%%%%%%%%%%%%%%%%%%%%%%%%%%%%%%%%%%%%%%%%%%%%%%%%
% \usepackage{layout}
% \makeatletter
% \renewcommand*{\lay@value}[2]{\strip@pt\dimexpr0.351459\dimexpr\csname#2\endcsname\relax\relax mm}
% \makeatother
%%%%%%%%%%%%%%%%%%%%%%%%%%%%%%%%%%%%%%%%%%%%%%%%%%%%%%%%%%%%%%%%%%
\def\Raumlaenge{400}
\def\Raumbreite{250}

\parindent=0pt
\parskip=1ex

\Stock{5106}{150} % o.k. % Gleise 5106: 150
\Stock{5107}{8}   % o.k. % Gleise 5107: 8 
\Stock{5100}{80}  % o.k. % Gleise 5100: 80
\Stock{5200}{37}  % o.k. % Gleise 5200: 37
\Stock{5120}{15}  % Gleise 5120: 15
\Stock{5214}{3}   % Dreiwegweichen
% 2 Bogenweichen links lang
% 1 Bogenweichen rechts lang
\Stock{5128}{1}   % 1 5128 Kreuzungsweiche
\Stock{5207}{2}   % % 2 Kreuzungsweichen 5207
\Stock{5203L}{1}  % 1 5203 L
\Stock{5118R}{1}  % 6 5119 R
\Stock{5119R}{6}  % 1 5118 L
\Stock{5204R}{4}  % 4 5204 R
\Stock{5118L}{3}  % 3 5118 L
% 5 Prellböcke

%%%%%%%%%%%%%%%%%%%%%%%%%%%%%%%%%%%%%%%%%%%%%%%%%
\begin{document}
\thispagestyle{empty}
\begin{center}
\begin{tikzpicture}[scale=0.060]
	\coordinate (lu) at (0,0);                      % Raum links
	\coordinate (lo) at (0,\Raumlaenge);            % Raum links
	\coordinate (ro) at (\Raumbreite,\Raumlaenge);  % Raum rechts
	\coordinate (ru) at (\Raumbreite,0);            % Raum rechts
	\coordinate (cu) at (120.5,\Raumlaenge);
	
	% \draw [name path=raum] \DebugCoord(lu) -- (lo)  -- (ro) -- \DebugCoord(ru) -- cycle;
	\draw ($(cu) + (0,10)$) node {Eisenbahn -- Otterswang};

	\draw [name path=anlage] (lu) -- (lo)  -- (ro) -- (ru) 
		-- ($(ru) - (100, 0)$) 
		-- ($(ru) - (100,-300)$) 
		-- ($(lu) - (-100,-300)$) 
		-- ($(lu) + (100,0)$) -- cycle ;

  \def\Angle{180}
  \coordinate (Start) at (50,10);
	\draw (Start)
			 \SavePosAngle{outer-0}
			 \Rep{3}{\RCurve{5200}}
			 \RSwitch[W1][O]{5204}
       \Straight{5106} 
			 \RCurve{5200}
			 \LCurve{5100}
			 \SavePosAngle{outer-1}
       ;
  \def\Angle{180}
  \draw ($(Start) +(0, 7.74)$)
			 \SavePosAngle{inner-0}
			 \Rep{3}{\RCurve{5100}}
			 \RSwitch[W2][O]{5204}
			 \XSwitch[W3][O]{5207}
			 \RCurve{5100}
			 \LCurve{5200}
			 \SavePosAngle{inner-1}
			 ;

   \RestPosAngle{inner-1}
	 \draw[color=red] (inner-1)
        \Up{2.00}
			 \Rep{12}{\Straight{5106}}
			 \Flat
			 % node [rotate=90, right] {H: \Height}
			 \SavePosAngle{inner-2}
			 ;
   
   \RestPosAngle{outer-1}
   \draw[color=red] (outer-1)
        \Up{2.00}
			 \Rep{12}{\Straight{5106}}
			 \Flat
			 node [rotate=90,above] {H: \Height}
			 \SavePosAngle{outer-2}
			 ;
   
   \RestPosAngle{outer-2}
   \draw[color=blue] (outer-2)
			 \Rep{3}{\RCurve{5200}}
			 \Rep{7}{\Straight{5106}}
			 \Straight{5107}
			 \Straight{5108}
			 \Rep{3}{\RCurve{5200}}
			 \Rep{6}{\Straight{5106}}
			 \Rep{8}{\RCurve{5200}}
			 \Rep{1}{\Straight{5106}}
			 \Rep{5}{\LCurve{5100}}
			 \SavePosAngle{outer-3}
			 ;

   \RestPosAngle{inner-2}
   \draw[color=blue] (inner-2)
			 \Rep{3}{\RCurve{5100}}
			 \Rep{7}{\Straight{5106}}
			 \Straight{5107}
			 \Straight{5108}
			 \Rep{3}{\RCurve{5100}}
			 \Rep{6}{\Straight{5106}}
			 \Rep{8}{\RCurve{5100}}
			 \Rep{1}{\Straight{5106}}
			 \Rep{5}{\LCurve{5200}}
			 \SavePosAngle{inner-3}
			 ;
   
   \RestPosAngle{outer-3}
   \draw[color=red] (outer-3)
   \Down{2.15}
			 \Rep{4}{\Straight{5106}}
			 \Flat
			 \SavePosAngle{outer-4}
			 node [below right] {H: \Height}
			 ;

   \RestPosAngle{inner-3}
   \draw[color=red] (inner-3)
   \Down{2.15}
			 \Rep{4}{\Straight{5106}}
			 \Flat
			 \SavePosAngle{inner-4}
			 ;

   \RestPosAngle{outer-4}
   \draw[color=green] (outer-4)
			 \Rep{2}{\LCurve{5100}}
			 \Rep{1}{\RCSwitch[B7][S]{5142}} 
			 \Rep{2}{\Straight{5106}}
			 ;

   \RestPosAngle{inner-4}
   \draw[color=green] (inner-4)
			 % Hier Außenkreis
			 \Rep{1}{\LCurve{5100}}      % 5100!
			 \Rep{1}{\LCSwitch[B7][O]{5141}}
			 \Rep{1}{\LCurve{5200}}
			 \Rep{2}{\Straight{5106}} 
			 ;
   
   \RestPosAngle{inner-0}
   \def\Angle{0}
   \draw (inner-0)
			 \Rep{3}{\LCurve{5100}}
			 \Rep{1}{\Straight{5106}}
			 \XSwitch[W4][B]{5207}   %% korrigieren
			 \LSwitch[W5][O]{5203}
			 \Rep{10}{\Straight{5106}}
			 \Rep{5}{\RCurve{5200}}
			 \Rep{2}{\Straight{5106}}
			 \Rep{1}{\RCurve{5200}}
			 \Rep{3}{\Straight{5106}}
			 \Rep{1}{\RCurve{5200}}
			 \Rep{1}{\Straight{5106}}
			 \Rep{1}{\Straight{5107}}
			 \Rep{1}{\Straight{5107}}
			 \Rep{1}{\LCurve{5100}}
			 \Rep{4}{\Straight{5106}}
			 \Rep{6}{\LCurve{5100}}
			 \Rep{14}{\Straight{5106}}
			 \Rep{1}{\Straight{5107}}
			 \Rep{1}{\Straight{5108}}
			 \Rep{1}{\Straight{5108}}
			 \Rep{3}{\LCurve{5100}}
			 \Rep{6}{\Straight{5106}}
			 \Rep{1}{\Straight{5108}}
			 \Rep{3}{\LCurve{5100}}
			 \Rep{2}{\Straight{5106}}
			 ;

   \RestPosAngle{outer-0}
   \def\Angle{0}
   \draw (outer-0)
			 \Rep{3}{\LCurve{5200}}
			 \LSwitch[W6][O]{5203}
			 \Rep{12}{\Straight{5106}}
			 \Rep{5}{\RCurve{5100}}
			 \Rep{2}{\Straight{5106}}
			 \Rep{1}{\RCurve{5100}}
			 \Rep{3}{\Straight{5106}}
			 \Rep{1}{\RCurve{5100}}
			 \Rep{1}{\Straight{5106}}
			 \Rep{1}{\Straight{5107}}
			 \Rep{1}{\Straight{5107}}
			 \Rep{1}{\LCurve{5200}}
			 \Rep{4}{\Straight{5106}}
			 \Rep{6}{\LCurve{5200}}
             \Rep{14}{\Straight{5106}}
			 \Rep{1}{\Straight{5107}}
			 \Rep{1}{\Straight{5108}}
			 \Rep{1}{\Straight{5108}}
			 \Rep{3}{\LCurve{5200}}
			 \Rep{6}{\Straight{5106}}
			 \Rep{1}{\Straight{5108}}
			 \Rep{3}{\LCurve{5200}}
			 \Rep{2}{\Straight{5106}}
			 ;

   \RestPosAngle{W5C}
   \draw (W5C)
			 \Rep{1}{\RCurve{5206}}
			 \Rep{6}{\Straight{5106}}
			 % \Rep{1}{\LCurve{5100}}
			 \TSwitch[W7][O]{5214}
   			 \RestPosAngle{W7L}
			 \Rep{1}{\RCurve{5206}}
			 \Rep{1}{\Straight{5108}}
			 ;
   \RestPosAngle{W4C}
   \draw (W4C)
			 \Rep{1}{\Straight{5106}}
			 \Rep{1}{\RCurve{5206}}
			 \Rep{6}{\Straight{5106}}
			 \Rep{1}{\LCurve{5100}}
			 \Rep{1}{\RCurve{5100}}
			 \Rep{1}{\Straight{5108}}
			 \Rep{1}{\Straight{5109}}
			 ;
\end{tikzpicture}
\end{center}

\newpage
\makeatletter

\TrackUse

\newpage

Raum: 

Winkel Dach 82 Grad

Breite Raum 370

Raum Länge 440

Tür 70

Strom: 80 vom Glas, 100 vom Dach

Gleise 5106: 150

Gleise 5107: 8 

Gleise 5100: 80

Gleise 5200: 37

Gleise 5120: 15

3 Dreiwegweichen

2 Bogenweichen links lang

1 Bogenweichen rechts lang

2 Kreuzungsweichen 5207

1 5203 L

1 5128 X - Kreuzungsweiche

6 5119 R

1 5118 L

4 5204 R

3 5118 L

5 Prellböcke

8 5102 1/2

\url{http://www.drahtkupplung.de/gtbhb2427.html}

\newpage
% \includegraphics[angle=90,width=1.0\linewidth,page=4]{0700.pdf} \\
% 
% \includegraphics[angle=90,width=1.0\linewidth,page=5]{0700.pdf} \\
% 
% \includegraphics[angle=90,width=1.0\linewidth,page=6]{0700.pdf} \\
% 
% \includegraphics[angle=90,width=1.0\linewidth,page=7]{0700.pdf} \\
% \includegraphics[width=\linewidth,page=8]{0700.pdf}
\newpage
\begin{tikzpicture}[scale=0.10]
   \draw (43.74,0) coordinate (C) arc [radius = 43.74, start angle=0, end angle=90] ;
   \draw (43.74,0) arc [radius = 36.00, start angle=0, end angle=30] ;
   \draw (36.00,0) arc [radius = 36.00, start angle=0, end angle=90] ;
\end{tikzpicture}
% \layout
\end{document}
